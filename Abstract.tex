\begin{abstract}
The design of Cyber-Physical Systems (CPSs) has become increasingly complex and challenging. The collaborative design is a potential approach to deal with the rise of the complexity of system design which attracts more and more attention. Collaborative design relied on Model-driven engineering (MDE) with involving several modeling languages (such as SysML, AADL, etc.) and using principles of separation of concerns as well as domain-specific languages (DSL). It makes stakeholders from diverse domains to work in a coordinated manner on different aspects of the system. It could help in reducing the gap between heterogeneous domains and making diverse expertise work together to produce a coherent and complete system. Therefore, there are requirements for collaborative design to allow modelers to work together. In this paper, we show how to use the coordinated metamodel approach in MDE as a systematic way to gather isolated domain models and cross-cutting concerns. In practice, we proposed a set of transformation operators to manipulating metamodels; it aims to enrich the capacities of the platform by blending different languages seamlessly, as well as perform verification and validation respectively at the high-level and early stage of development. We illustrate our approach with an experimental of (engine??) design processes represented as scenarios.

\end{abstract} 
