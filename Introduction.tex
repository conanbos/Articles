\section{introduction}
The conception of multi-view design has been proposed to respond to currently complex system's requirements. It relied on Model-based approach (MBA) and involved several modeling languages (such as SysML, AADL, etc.). However, we are not able to integrate all of them into a mono design platform. Hence, we proposed a new approach to blending them seamlessly and extended their capabilities without extending development platform, that we named co-design operation. Firstly, we are established in engineering framework --Capella which adopts SysML as a modeling language. Furthermore, we proposed a set of transformation operators that can construct an interpretation system to automatically translate  Capella designs into AADL develop environment (OASTE), and then it can refine models and performance verification and validation respectively. We illustrate our approach with an experimental of (engine??) design processes represented as scenarios.


Nowadays, A system becomes more and more complex which involved a number of aspects, especially, the Cyber-physical systems (CPSs) which is 


object-oriented  meta-languages such as MOF (Meta-Object Facility) are increasingly used to  specify domain-specific languages in  the   model-driven   engineering   community.   However,   these   meta-languages focus on structural specifications and have no built-in support for specifications 
of  operational  semantics.  In  this  paper  
we  explore  the  idea  of  using  aspect-
oriented modeling to add precise action specifications with stat
ic type checking 
and  genericity  at  the  meta  level,  a
nd  examine  related  issues  and  possible  
solutions. We believe that such a comb
ination would bring significant benefits 
to   the   community,   such   as   the   specification,   simulation   and   testing   of   
operational   semantics   of   
metamodels.   We   present   requirements   for   such   
statically-typed meta-languages and ra
tionales for the aforementioned benefits. 