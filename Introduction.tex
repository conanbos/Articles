\section{introduction}\label{sec:intro}
The design of Cyber-Physical Systems (CPSs) has become increasingly complex and challenging. The developer should face numerous aspects, and each aspect has their problem space and characteristics. 

As the Model-Based Engineering (MBE) has emerged as a key set of technologies to engineer the complex system. Experts thus usually used models and metamodels as the common countermeasure to capturing and solving the problems. In recent years, a number of modeling languages target CPSs have been standardized, but, to the best of our knowledge, none of them provide the full range required to deal effectively with the kinds of CPSs's complex that we have encountered. Some of them, such as Systems Modeling Language (SysML)~\cite{Group:2017vva}, a more general-purpose language, focus on the “big picture” requirements, functional and architectural views, whereas others, such as Architecture Analysis and Design Language (AADL)~\cite{Anonymous:aEmc45af} address the more detailed platform-oriented and physical aspects of such systems.  

Since Cyber-Physical Systems interact with surrounding physical environment frequently, and discrete, continuous and hybrid behaviors describe the physical part, fox example, the physical symptom used DEs or PDEs (Partial Differential Equations) to express. A more general case is the hybrid system which is the continuous dynamics of interactions between a control system and its environment, thus Ahmad et al~\cite{Ahmad:2014:HAA:2692956.2663178} proposed a hybrid annex of AADL to intensify AADL's capability to describe hybrid systems. In this paper, we also considered HA (hybrid annex) as a part of the target model. 

Although AADL was a sophisticated analysis language and existing IDE tools such as OSATE~\cite{osate2ref} which can describe and verify both functional and non-functional properties of AADL models, it still insufficient to cater comprehensive system design and the need of abilities for engineering~\cite{behjati2011extending}. Therefore, AADL must be used with the upstream modeling language such as SysML and supporting engineering environment like  Capella/Arcadia~\cite{capella2014}\cite{AModelBasedEngine:JlLHIqkz}. 

Capella platform and its method ARCADIA (\textbf{ARC}hitecture \textbf{A}nalysis and \textbf{D}esign \textbf{I}ntegrated \textbf{A}pproach) are to some extent a SysML-like solution to design the architecture of complex systems using models. However, unlike SysML, it appears to be a domain-specific language (DSL) which was preferred in order to ease appropriation by all stakeholders, usually not familiar with general-purpose, generic languages such as UML or SysML. Rather than having modeling “experts” owning the model on behalf of systems engineers, the ultimate goal of Arcadia is to have systems engineers pay more attention to global system design. 

It is seldom the case that one development platform or single language can adapt to all aspects with assumption one-size-fits-all. Due to the engineers involved various modeling languages to different model domains respectively, and the engineers who from the different domain, each one has domain-specific tools. It does not only result in the proliferation of languages and its extensions for describing a variety of CPSs's aspects but also make the design complexity of CPSs increase. Meanwhile, the gaps between languages and platforms brought additional problems such as coherent and consistent problems, which is exposed at integration and simulation stages and further augment the complexity, make it skyrocketing. 
 
To reduce the gaps and eliminate inconsistencies between SymML--like tool Capella/Arcadia and architecture design language AADL and its hybrid annex (HA), also to benefit from both of their powerful capabilities and strong points, we present a suite of evolving combining approach. In despite of existing a lot of studies on the combining SysML and AADL~\cite{de2012combining} or on the extending SysML with AADL~\cite{behjati2011extending}. Differ from above studies; our method dedicates to smoothly combine engineering platform Capella/Arcadia, AADL and its annex, in this way, one could design global system at a high level and then seamlessly refine the models within AADL and its annex. That can also properly extend Arcadia's capability, and make it can perform analysis of discrete behavior variation and continuous environmental domain.    


This paper presents an approach was to characterize the discrete and continuous properties of CPSs by models and abstract metamodels, respectively. Then merge the two metamodels to generate new metamodel at high abstract level. The engineer can use this generated metamodel to create kaleidoscopic instance models which can be used to further analysis (e.g., timing, safety). In this way,  it allows the original engineering platform is capable of scaling to a wider range of capabilities through the use of relevant languages. Compared with similar studies, our approach makes \textbf{\textit{three following major contributions}}:

\begin{itemize}
    \item A proper subset of AADL and its annex are chosen as the target metamodel including the thread (scheduling, dispatching and execution), port connections, and hybrid behavior. ii) Partial components of Arcadia (conform to SysML) are abstracted as the metamodel. Both above metamodels are defined formally.
    \item A set of relationships (e.g., transforming, creating, ignoring), semantic definitions and corresponding rules are provided to help the integration engineer custom-tailor metamodel they needed. 
    \item we implement the transformation process among metamodels and practice our approach with a toolchain.   
\end{itemize}

The rest of this paper is organized as follows: in the next section, we present the state-of-the-art of engineering models transformation briefly. Then the summary of our approach process is presented, as well as the metamodels, and its formal definitions are given in the section~\ref{sec:app}. Section \ref{sec:trans} gives a set of relationships, semantics and corresponding rules. Section \ref{sec:imp_cs} describes the implementation and using a case study of train traction controlling systems to demonstrate architecture and secluding analysis. Conclusion and future work presented in the last section.
