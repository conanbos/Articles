\section{introduction}
The design of Cyber-Physical Systems (CPSs) has become increasingly complex and challenging. %Cyber-Physical Systems (CPSs) were a synthesis, and a complex system has become consensus. 
The developer should face numerous aspects, and each aspect has their features and characteristics, as well as special modeling languages and platforms. The collaborative design is a potential approach to deal with the rise of the complexity of system design, attracting more and more attention from the industry and the scientific community. The collaborative design relied on Model-driven engineering (MDE) with involving several modeling languages (such as SysML, AADL, etc.) and using principles of separation of concerns as well as domain-specific languages (DSL). It makes stakeholders from diverse domains to work in a coordinated manner on different aspects of the system. It could help in reducing the gap between heterogeneous domains and making diverse expertise work together to produce a coherent and complete system. However, we are not able to integrate all of them into a mono design platform. Hence, we proposed a new approach to blending them seamlessly and extended their capabilities without extending develop platform. 


Our approach works at the high-level and early stage of development. 


Therefore, there is a need for collaborative platforms to allow modelers to work together. We proposed a new approach to blending different languages seamlessly by manipulating metamodel with a set of transformation operators at the high-level and early stage of development. The process starts from a SysML system model, developed according to the platform-based design (PBD) paradigm, in which a functional model of the system (SW) is paired to a model of the execution platform (HW). Subsystems are refined as AADL models. In turn, AADL models are implemented as code and transfer to model-checker to perform verification and validation respectively. We illustrate our approach with an experimental of (engine??) design processes represented as scenarios.





The conception of multi-view design has been proposed to respond to currently complex system's requirements. It relied on Model-driven engineering (MDE) with involving several modeling languages (such as SysML, AADL, etc.) and using principles of separation of concerns as well as domain-specific languages (DSL) to make stakeholders from diverse domains to work in a coordinated manner on different aspects of the system. It could help in reducing the gap between heterogeneous domains and making diverse expertise work together to produce a coherent and complete system. Therefore, there is a need for collaborative platforms to allow modelers to work together. We proposed a new approach to blending different languages seamlessly by manipulating metamodel on the high-level and early stage of development. Firstly, we are established in engineering framework (Capella) which adopts SysML as a modeling language. Furthermore, we proposed a set of transformation operators that can construct an interpretation system to translate designs into AADL develop environment (OASTE) automatically, and then it can refine models and performance verification and validation respectively. We illustrate our approach with an experimental of (engine??) design processes represented as scenarios.


The process start from a SysML system model, developed according to
the platform-based design (PBD) paradigm, in which a functional model
of the system is paired to a model of the execution platform. Subsystems
are refined as Simulink models or hand coded in C++. In turn, Simulink
models are implemented as software code or firmware on FPGA, and an
automatic generation of the implementation is obtained. Based on the
SysML system architecture specification, our framework drives the gener-
ation of Simulink models with consistent interfaces, allows the automatic
generation of the communication code among all subsystems (including
the HW-FW interface code).

The conception of multi-view design has been proposed to respond to currently complex system's requirements. It relied on Model-based approach (MBA) and involved several modeling languages (such as SysML, AADL, etc.). However, we are not able to integrate all of them into a mono design platform. Hence, we proposed a new approach to blending them seamlessly and extended their capabilities without extending development platform, that we named co-design operation. Firstly, we are established in engineering framework --Capella which adopts SysML as a modeling language. Furthermore, we proposed a set of transformation operators that can construct an interpretation system to automatically translate  Capella designs into AADL develop environment (OASTE), and then it can refine models and performance verification and validation respectively. We illustrate our approach with an experimental of (engine??) design processes represented as scenarios.


Nowadays, A system becomes more and more complex which involved a number of aspects, especially, the Cyber-physical systems (CPSs) which is 


%developed according to the platform-based design (PBD) paradigm, in which a functional model of the system (SW) is paired to a model of the execution platform (HW). Subsystems are refined as AADL models. In turn, AADL models are implemented as code and transfer to model-checker to perform verification and validation respectively. We illustrate our approach with an experimental of (engine??) design processes represented as scenarios.


 
%The development of complex software-intensive systems requires stakeholders from diverse domains to work in a coordinated manner on different aspects of the system. Model-driven engineering (MDE) helps in reducing the gap between heterogeneous domains using principles of separation of concerns, automatic generation and domain-specific languages (DSL). MDE is thus a potential solution to help develop systems collaboratively. In MDE, stakeholders work on models in order to design, transform, simulate, and analyze systems. Teams of stakeholders with varying expertise work together to produce a coherent and complete system. Therefore, there is a need for collaborative platforms to allow modelers to work together.
%we present the different research projects we have been working on in the topic of collaborative modeling in MDE. After laying a set of necessary requirements I address several topics. We will look at the data structures needed to ensure near real-time collaboration. We will see how multi-view modeling is essential to let users work on different aspects of the system concurrently. Having users from different domains and expertise, we show how to adapt the modeling environment specifically to the needs and habits of the user. We will discuss automatic generation of restrictive modeling IDEs that can support all sorts of user interaction modes.